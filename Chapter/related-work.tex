\chapter{Related Work}\label{cap:related-work}
\thispagestyle{empty}

	This chapter presents the state-of-the-art on the main topics related to the proposed work. To elucidate the importance and evolving role of peering infrastructures in the Internet topology, in Subsection~\ref{sec:rel-work-peering-infra} we discuss the most relevant studies that analyze the operational and internal characteristics of these infrastructures. 
	%In Subsection~\ref{sec:rel-work-relat-as}, we present the efforts to identify the type of peering relationships between ASes and how this classification can increase knowledge about aspects of the networks, such as geolocation. 
	%Mapping of peering interconnections and topology elements to physical locations improve understanding of network performance and resilience. The main related work tries to improve the accuracy of this mapping and are discussed in Subsection~\ref{sec:rel-work-mapping-peer} and \ref{sec:rel-work-mapping-topo-elem}, respectively.
	In Subsection~\ref{sec:rel-work-mapping-peer} and \ref{sec:rel-work-mapping-topo-elem}, we present the efforts to improve the accuracy of mapping peering interconnections and topology elements to physical locations.

	\section{Peering infrastructures and operation}
	\label{sec:rel-work-peering-infra}

	The role of peering structures is increasingly essential for the exchange of inter-domain traffic on the Internet. Studies reveal that existing Internet Exchange Points are responsible for transferring amounts of data similar to
	Tier 1 Internet Service Provider (ISPs)~\cite{Ager:2012}. Chatzis et al. ~\cite{Chatzis:2013:BUL:2504730.2504746} report that one of the largest European IXPs can observe traffic from a large portion of the Internet, including 42K+ routed ASes, almost all 450K+ routed prefixes and around a quarter billion IP addresses from all the countries around the globe. 

	Richter et al.~\cite{Richter:2014} point that the membership rates of ASes connecting to these infrastructures have a growth of 10-20\% annually and a growth in traffic rates of 50-100\% per year. Kotronis et al.~\cite{Kotronis:2015:IPI:2745844.2745877} show that about 40\% of IP prefixes advertised on the Internet can be reached directly from around 5 IXPs. Besides, despite the focus to deploy peering infrastructures in Europe and USA~\cite{Chatzis:2013}, studies reveal that developing regions as Latin America, and Africa are recently increasing the adoption of IXPs to enhance network performance~\cite{DissectingBrazilianIXP, Fanou:2017:ICC:3131365.3131394}. 

	The mentioned studies show that these infrastructures are becoming available worldwide and have a broad vision of the Internet. Thus, they present an potential opportunity for generating rich geolocation inferences and improve knowledge about the Internet topology.

	%\section{Relationship between ASes}
	%\label{sec:rel-work-relat-as}

	%\cite{Giotsas:2014:ICR:2663716.2663743, Giotsas:2013, Luckie:2013:RCC:2504730.2504735}

	\section{Mapping of peering interconnections and infrastructures}
	\label{sec:rel-work-mapping-peer}

	Measuring and mapping the Internet at AS-level is valuable to understand the underneath structure of the topology. However, it considerably abstracts rich information about connectivity between networks at the Internet. Accurate knowledge of interconnection geolocation facilitates network troubleshooting, outage detection, and attack diagnosis. Recent efforts attempt to infer peering matrices (i.e., who peers with whom at which IXP) and map interconnections to physical locations where they occur. 

	Augustin et al.~\cite{Augustin:2009:IM:1644893.1644934} proposes a method to detect IXPs, identify IXP-specific peering matrices and better understand the IXP substrate of the Internet’s AS-level ecosystem. The mechanism detected 278 IXPs and discovered the existence of about 44K IXP-related peering links. However, the method has a very high cost of time and active measurements. Kotronis et al.~\cite{nomikos2016traixroute} extend the traceroute tool with the capability of inferring if and where an IXP was crossed. Results show that approximately one out of five paths crosses an IXP and that IXP-paths usually cross no more than a single IXP. 

	Giotsas et al.~\cite{Giotsas:2015:MPI:2716281.2836122} propose an algorithm to infer the physical interconnection facility where an interconnection occurs among all possible candidates. The methodology provides accurate results but is unable to scale to large scenarios involving several colocation facilities, IXPs and ASes, given a large amount of active probing resources needed. The initial process of mapping networks and IXPs to facilities, required by the methodology, is manual and time-consuming to be developed/updated.

	%\cite{Giotsas:2015:MPI:2716281.2836122} propose an algorithm to infer the physical interconnection facility where an interconnection occurs among all possible candidates. Uses data from PeeringDB, PCH, IXP, ASes and Network Operating Centers (NOCs) sites and lists provided in regional consortia of IXPs to build an initial mapping between AS, IXP, and facilities.
	%Measurements from RIPE Atlas, Looking Glass, iPlane and CAIDA Ark.
	
	%Methodology provides accurate results for the interfaces that resolve to a facility in a low number of iterations. Key insight is that the type of peering for an interconnection sufficiently constrains the number of candidate facilities to identify the specific facility where a given interconnection occurs.
	
	%Unable to scale to large scenarios involving several colocation facilities, IXPs, and ASes, given a large amount of active probing campaigns needed. Initial process of mapping networks and IXPs to facilities, required by the methodology, is extremely manual and time-consuming to be developed/updated. Methodology is very sensitive to missing or incorrect information and could yield inaccurate results. 

	\section{Mapping of topology elements}
	\label{sec:rel-work-mapping-topo-elem}


	The geolocation of network elements, mainly mapping the IP addresses of routers to physical locations with precision is a crucial task. Precise knowledge of router geolocation helps to detect BGP threats, estimate the geographic presence of ASes and customize content delivery. It is possible to geolocate IP addresses through public or commercial databases, delay-based or DNS-based methods.

	Gharaibeh et al.~\cite{Gharaibeh:2017:LRG:3131365.3131380} compare router geolocation coverage and reliability in four popular geolocation databases. The authors show that despite having a  high coverage at country-level, databases are not accurate in geolocating routers at neither country- nor city-level, even if they agree significantly among each other. The work of Huffaker et al.~\cite{Huffaker:2014:DDR:2656877.2656879} and Scheitle et al.~\cite{8002903} propose methods to geolocate routers based on geography-related strings in hostnames and validate the results with active measurements from different decentralized probes. Despite showing accurate results, the scope of both proposals is restricted since only a small subset of routers have apparent geographic hints in their DNS names.

	The work mentioned in the two previous subsections (\ref{sec:rel-work-mapping-peer} and \ref{sec:rel-work-mapping-topo-elem}) shows vast potential and opportunity to improve the generation of geolocation inferences. Current solutions either rely on a large number of active measurements and decentralized probes to be able to achieve accurate results or provide rich inferences for just a subset of topology elements.
	%\cite{Gharaibeh:2017:LRG:3131365.3131380, Huffaker:2014:DDR:2656877.2656879, 8002903}

	%\cite{Gharaibeh:2017:LRG:3131365.3131380} shows that current commercial and public databases are not accurate in geolocating router at neither country- nor city-level. \cite{Huffaker:2014:DDR:2656877.2656879, 8002903} propose methods to accurately geolocate routers based on geography-related strings in hostnames and validate the results with active measurements from different decentralized probes. However, their scope is restricted since only a small subset of routers have apparent geographic hints in their DNS names.

	%The work of \cite{Giotsas:2017:DPI:3098822.3098855} combines location-tagging BGP Communities with a colocation map to infer the location of outages. Map is done similarly as in \cite{Giotsas:2015:MPI:2716281.2836122}. BGP communities have no standardized semantics and are only employed by half of the BGP paths announced on the Internet, which could lead to an incorrect and incomplete view of the infrastructure.

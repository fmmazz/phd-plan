\chapter{Contextualization of Thesis Project}\label{cap:contextualization}
\thispagestyle{empty}

A tese de doutorado proposta neste plano está associada aos trabalhos em Internet do Futuro realizados pelo grupo de Redes de Computadores da UFRGS, em particular aos seguintes projetos:

\begin{itemize}
    \item GT-IPÊ analytics: Transformando dados brutos de monitoramento em informações valiosas ao gerenciamento de rede. Este GT busca o desenvolvimento de um sistema que analisa os dados coletados pelos monitoramentos existentes na rede IPÊ. O objetivo é fornecer diversas informações relevantes para amparar a gerência de operações, engenharia de tráfego e o planejamento da rede, tais como correlações e categorizações automatizadas, perfis de uso sobre diferentes aspectos, filtragem e manipulação interativas sobre os dados, estatísticas sobre a eficiência da engenharia de tráfego e possibilidade de análise do impacto de mudanças. Para tanto, o sistema utilizará ferramentas de manipulação de grandes volumes de dados e técnicas de aprendizado de máquina.
    \item Azure4Research: Em busca de um melhor projeto e manutenção de redes. Avanços recentes na área de redes fizeram tanto a indústria como a acadêmia mudarem para um novo paradigma de gerenciamento de redes, as Redes Definidas por Software. Este paradigma emparelhou seus benefícios com desafios no projeto e manutenção de redes. Esta proposta busca otimizar o projeto e a manutenção dessas redes. \textcolor{red}{}A plataforma Microsoft Azure pode auxiliar neste pesquisa em duas formas: ($i$) permitindo simulações e experimentos de larga escala, com diferentes tipos de topologias (em particular, com as topologias do estado da arte na área de \textit{data centers}); e ($ii$) permitindo o estudo e desenvolvimento de modelos complexos de otimização (com objetivos variados), os quais consideram múltiplas propriedades em sua formulação.}
\end{itemize}

O grupo atualmente conta com 3 doutorandos nesta área. Ademais, o grupo de redes mantém contato com professores de diversos grupos internacionais, gerando oportunidades para curso de doutorado sanduíche.

\chapter{Introduction}\label{cap:introduction}
\thispagestyle{empty}

	%\textbf{Context.} Peering infrastructures, such as colocation facilities and Internet Exchange Points (IXPs) are increasingly deployed all over the world, supporting a growing number of network members and peering interconnections \cite{Giotsas:2017:DPI:3098822.3098855}. Colos are physical locations that provide essential infrastructure as power, space and supports interconnection of networks. IXPs are physical infrastructures and provide a shared switching fabric where participating networks can interconnect their routers. \cite{Giotsas:2015:MPI:2716281.2836122}
	\textbf{Context.} The flattening of the Internet causes inter-domain traffic to bypass transit providers and flows directly between edge networks~\cite{Labovitz:2010:IIT:1851182.1851194}. The need for guarantees in traffic delivery with efficiency and resilience~\cite{Yap:2017:TEO:3098822.3098854, Schlinker:2017:EEE:3098822.3098853, marcos:2018:dynamix} drives Autonomous Systems (ASes) to interconnect at peering infrastructures, such as colocation facilities and Internet Exchange Points (IXPs)~\cite{Giotsas:2015:MPI:2716281.2836122}. These infrastructures simplify interconnection among networks within a region, improving network performance due to lower latency and routing efficiencies~\cite{Chatzis:2013}.

	%Brazil peering infrastructure maintains the largest set of public IXPs in a single country and is among the world’s top ten IXPs in terms of traffic.~\cite{DissectingBrazilianIXP}. Currently, there are 31 public IXPs deployed in all regions~\cite{IXbr}. Access to privileged data (i.e., flows samples and BGP information) of the major IXPs of the IX.br Project (e.g., IX-SP and IX-RS) leaves large room to explore measurement studies. 

	\textbf{Motivation.} A key point to achieve the goals of network performance and resilience is to understand what is occurring in the physical topology of the Internet. Even though mapping the Internet with precision is a challenging problem due to the growing complexity of networking infrastructure and security and commercial sensitivities~\cite{Giotsas:2015:MPI:2716281.2836122}, it is an essential task. Getting geolocation inferences about peering interconnection (e.g., multilateral agreements~\cite{Giotsas:2013}) and topology (e.g. routers and switches~\cite{8002903,Huffaker:2014:DDR:2656877.2656879}) can enhance the understanding of the dynamics of Internet traffic~\cite{marcos:2018:dynamix}, improve traffic delivery~\cite{Yap:2017:TEO:3098822.3098854, Schlinker:2017:EEE:3098822.3098853} and infrastructure planning~\cite{Calder:2013:MEG:2504730.2504754} and increase responsiveness to outages and attacks~\cite{Giotsas:2017:DPI:3098822.3098855}.

	Recent work focus either in generating geolocation inferences about peering interconnections or topology elements. In the first group, \cite{Giotsas:2015:MPI:2716281.2836122, Augustin:2009:IM:1644893.1644934} provide accurate results to geolocate interconnections, but can not scale to large scenarios with several facilities, IXPs, and ASes, given a large amount of decentralized active probing campaigns needed. For the latter, \cite{Gharaibeh:2017:LRG:3131365.3131380} show that current commercial and public databases are not accurate in geolocating router at neither country- nor city-level. \cite{Huffaker:2014:DDR:2656877.2656879, 8002903} proposes methods to accurately geolocate routers based on geography-related strings in hostnames and validate the results with active measurements from different decentralized probes. However, their scope is restricted since only a small subset of routers have apparent geographic hints in their DNS names.

	In this context, IXPs emerge as potential candidates to improve the mapping of the Internet. These infrastructures play a global role, since they have large volume of information on data and control plan, see traffic from a significant fraction of the Internet~\cite{Chatzis:2013:BUL:2504730.2504746} and are increasingly being deployed all over the world, supporting a growing number of network members and peering interconnections~\cite{Giotsas:2017:DPI:3098822.3098855}.
	
	%Expand existing results would provide a more scalable and more accurate interconnection mapping which would help to pinpoint outages and attacks, troubleshoot network problems, track traffic flows and improve the resilience of interconnections more precisely.

	%Investigate if leveraging IXPs as scalable vantage points, performing measurements campaigns from IXP to the outside Internet, and using available privileged data (i.e., flows samples and BGP information) can improve the interconnection mapping to facilities and create a broader, more accurate and scalable methodology which would be used to enhance network infrastructure and safety. Looking Glass has some servers colocated with IXPs that can be used as vantage points. However, they are not scalable since they are not appropriate for scanning a large range of addresses due to probing limitations~\cite{Giotsas:2015:MPI:2716281.2836122}.

	\textbf{Proposal.} This work seeks to investigate the potential of IXPs as anchors to improve the Internet mapping and generate geolocation inferences of peering interconnections and topology elements. Our goal is to develop a precise and systematic approach to increase geolocation knowledge while making it scalable, automatized and low-cost. First, we aim to enhance geolocation inferences in developing regions such as Latin America, which show rich, but weakly examined, peering infrastructures~\cite{IXbr, DissectingBrazilianIXP}, following recent work~\cite{10.1007/978-3-319-15509-8_4, Fanou:2017:ICC:3131365.3131394}. Next, we aim to extend our methodology to other worldwide available IXPs and develop a better understanding of the geolocation characteristics of these infrastructures in the Internet Topology.

	%\textbf{Expected contributions.} We expect that leveraging IXPs as scalable vantage points, given their global role seeing traffic from a large fraction of the Internet~\cite{Chatzis:2013:BUL:2504730.2504746}, and using available privileged data (i.e., flows samples and BGP information) will enable higher visibility and knowledge about the geolocation of peering interconnections, and will allow creating a broader and more accurate interconnection mapping to facilities. Having a precise and scalable methodology to map peering interconnections to facilities could improve the understanding of interconnections efficiency (interconnection in different facilities in the same area could decrease the number of hops or latency) and level of peering interconnection redundancy.

	\textbf{Expected contributions.} We expect that leveraging IXPs as scalable vantage points will enable higher visibility and knowledge about the geolocation of peering interconnections and network elements. The broader vision will allow us to identify gaps in existing physical and measurements (BGP) infrastructures, observe how the connectivity between ASes varies in developing areas (e.g., Latin America) and across different regions, increase knowledge about redundancy, and investigate approaches to identify IXPs not present in public databases. 
	Recent events~\cite{routerDMV, routerUnited} show how failure in topology elements can delay or even cause disruption in services, affecting thousands of people. Mapping network topology and peering interconnections precisely could help to troubleshoot and diminish the time to repair similar problems.

	\textbf{Outline of the proposal.} Chapter~\ref{cap:background} provides background and terminology. Chapter~\ref{cap:related-work} presents the state-of-the-art and the main related work. In Chapter~\ref{cap:proposal}, we present the proposal work. Chapter~\ref{cap:expected-results} provides the expected results and main contributions. In Chapter~\ref{cap:methodology}, we show the expected methodology, set of steps and schedule. Finally, in Chapter~\ref{cap:coursework}, we present the expected coursework of the Ph.D. course.



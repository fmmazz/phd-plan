\chapter{Introduction}\label{cap:introduction}
\thispagestyle{empty}

	%\textbf{Context.} Peering infrastructures, such as colocation facilities and Internet Exchange Points (IXPs) are increasingly deployed all over the world, supporting a growing number of network members and peering interconnections \cite{Giotsas:2017:DPI:3098822.3098855}. Colos are physical locations that provide essential infrastructure as power, space and supports interconnection of networks. IXPs are physical infrastructures and provide a shared switching fabric where participating networks can interconnect their routers. \cite{Giotsas:2015:MPI:2716281.2836122}
	This document presents a research plan in the context of Internet measurements, to be developed during the PhD, in cooperation with Kimberly Claffy\footnote{\url{http://www.caida.org/~kc/}} and Bradley Huffaker\footnote{\url{http://www.caida.org/~bhuffake/}}. It follows an ongoing collaboration with the same group at CAIDA/UCSD, through the PhD student Lucas Muller. 

	\textbf{Context.} Recently, Autonomous Systems (ASes) are interconnecting at peering infrastructures, such as Internet Exchange Points (IXPs) and colocation facilities~\cite{Giotsas:2015:MPI:2716281.2836122} to achieve efficient and resilient traffic delivery~\cite{Yap:2017:TEO:3098822.3098854, Schlinker:2017:EEE:3098822.3098853, marcos:2018:dynamix}. These infrastructures simplify interconnection among networks within a region, improving network performance with lower latencies and better routing efficiency (e.g., fewer AS hops for end-to-end paths)~\cite{Chatzis:2013}. The possibility for direct interconnection between ASes allows inter-domain traffic to bypass transit providers and flow directly between edge networks, flattening the Internet’s hierarchical structure~\cite{Labovitz:2010:IIT:1851182.1851194}.

	%Brazil peering infrastructure maintains the largest set of public IXPs in a single country and is among the world’s top ten IXPs in terms of traffic.~\cite{DissectingBrazilianIXP}. Currently, there are 31 public IXPs deployed in all regions~\cite{IXbr}. Access to privileged data (i.e., flows samples and BGP information) of the major IXPs of the IX.br Project (e.g., IX-SP and IX-RS) leaves large room to explore measurement studies. 

	\textbf{Relevance of geolocation information.} 
	Understanding the developments in the Internet topology is crucial to achieving the goals of network performance and resilience and developing the Internet cartography. Such a task is challenging due to the growing complexity of networking infrastructure and commercial aspects~\cite{Giotsas:2015:MPI:2716281.2836122}. Geolocation inferences about peering interconnection (e.g., multilateral agreements~\cite{Giotsas:2013}) and topology elements (e.g. router IP addresses~\cite{8002903,Huffaker:2014:DDR:2656877.2656879}) can enhance the understanding of the dynamics of Internet traffic, improve traffic delivery and infrastructure planning~\cite{Calder:2013:MEG:2504730.2504754}, and increase responsiveness to outages and attacks~\cite{Giotsas:2017:DPI:3098822.3098855, marcos:2018:dynamix}. 

	%Developing areas such as Latin America show rich network topology. However, these areas are poorly investigated, and knowledge of network infrastructure is limited and inaccurate as current geolocation bases, and measurements infrastructures tend to focus in developed regions such as Europe and US. 

	\textbf{Current limitations.} Existing methodologies enable geolocation inferences about peering interconnections and topology elements, but fail to provide accurate results in a scalable way. There is a trade-off between accuracy and scalability, and no current active measurement approach can provide both at the same time, regardless whether active or passive. Active  geolocation methodologies can provide accurate results, but tend to require higher number of vantage points (i.e., hosts with known locations) from measurement infrastructures such as Ark~\cite{ark} or RIPE Atlas~\cite{ripeatlas} to geolocate router and hosts targets on the Internet via active probing. They normally generate a large volume of incoming traffic to the targets, produce a massive amount of collected data and require a significant amount of time for collecting and creating inferences, hampering the scalability.  Passive approaches (e.g., geolocation databases), instead, offer information that is readily available for users. However, this information is obtained by combining hostname hints, domain registry information, and other heuristics that can help to obtain geolocation knowledge. Generally, the methodology of these approaches is proprietary, which hinders reproducibility and makes the geolocation conclusions to be unreliable and likely to be inaccurate.

	%Recent efforts can be separated into two groups. One focuses on generating geolocation inferences about peering interconnections, while the other aim attention at getting the physical location of topology elements. In the former case, \cite{Giotsas:2015:MPI:2716281.2836122, Augustin:2009:IM:1644893.1644934} provide accurate results to geolocate interconnections, but cannot scale to large scenarios with several ASes, because of the large amount of decentralized active probing campaigns needed. For the latter group of related papers, \cite{Gharaibeh:2017:LRG:3131365.3131380} shows that current commercial and public databases are not accurate in geolocating router at neither country- nor city-level. \cite{Huffaker:2014:DDR:2656877.2656879, 8002903} propose methods to accurately geolocate routers based on geography-related strings in hostnames and validate the 24results with active measurements from different decentralized probes. However, their scope is restricted since only a small subset of routers have apparent geographic hints in their DNS names.

	\textbf{IXPs}, in this context, emerge as potential candidates to improve the mapping of the Internet, allowing both active measurements and passive analysis of collected data. These infrastructures play a global role in the Internet topology because carry traffic from a significant fraction of the Internet and generate a large volume of control plane information~\cite{Chatzis:2013:BUL:2504730.2504746}. They are increasingly being deployed all over the world, supporting a growing number of network members and peering interconnections~\cite{Giotsas:2017:DPI:3098822.3098855}. The available data and growing interconnection of ASes in IXPs enable higher visibility and knowledge about the network topology, showing potential in generating geolocation inferences.
	
	%Expand existing results would provide a more scalable and more accurate interconnection mapping which would help to pinpoint outages and attacks, troubleshoot network problems, track traffic flows and improve the resilience of interconnections more precisely.

	%Investigate if leveraging IXPs as scalable vantage points, performing measurements campaigns from IXP to the outside Internet, and using available privileged data (i.e., flows samples and BGP information) can improve the interconnection mapping to facilities and create a broader, more accurate and scalable methodology which would be used to enhance network infrastructure and safety. Looking Glass has some servers colocated with IXPs that can be used as vantage points. However, they are not scalable since they are not appropriate for scanning a large range of addresses due to probing limitations~\cite{Giotsas:2015:MPI:2716281.2836122}.

	\textbf{Proposal.} In this plan, we will investigate the potential of IXPs as powerful vantage points to improve the Internet cartography. We aim to produce a hybrid approach that combines the advantages of both active (i.e., accuracy) and passive (i.e., scalability) solutions while being reproducible. Our method intends to correlate data from active measurements performed from inside of the IXP with analysis of already existing information from control and data planes (e.g., BGP) to produce geolocation knowledge about peering interconnections and network elements. 

	Our goal is to develop a methodology that is, at the same time, \emph{accurate}, \emph{scalable} and \emph{reproducible}. More specifically, it must be: ($i$) \emph{accurate} by providing valid, correct and reliable geolocation conclusions; ($ii$) \emph{scalable} by using few but key vantage points and generating little processing and networking overheads; and ($iii$) \emph{reproducible} by not relying in complex and ad-hoc solutions and being easily reproducible by researchers and network operators. We plan to deploy and validate our methodology using the Brazilian IX ecosystem, to which we have access.

	%First, we aim to enhance geolocation inferences in developing regions such as Latin America, which show rich, but weakly examined, peering infrastructures~\cite{IXbr, DissectingBrazilianIXP}, following recent work~\cite{10.1007/978-3-319-15509-8_4, Fanou:2017:ICC:3131365.3131394}. Next, we aim to extend our methodology to other worldwide available IXPs and develop a better understanding of the geolocation characteristics of these infrastructures in the Internet topology.

	%\textbf{Expected contributions.} We expect that leveraging IXPs as scalable vantage points, given their global role seeing traffic from a large fraction of the Internet~\cite{Chatzis:2013:BUL:2504730.2504746}, and using available privileged data (i.e., flows samples and BGP information) will enable higher visibility and knowledge about the geolocation of peering interconnections, and will allow creating a broader and more accurate interconnection mapping to facilities. Having a precise and scalable methodology to map peering interconnections to facilities could improve the understanding of interconnections efficiency (interconnection in different facilities in the same area could decrease the number of hops or latency) and level of peering interconnection redundancy.

	\textbf{Expected contributions.} We believe that leveraging IXPs as scalable vantage points will provide greater visibility and understanding about the geolocation of peering interconnections and network elements. The proposed work expects to generate the following benefits: ($i$) improvement on the mapping methodologies of peering interconnections and topology elements; ($ii$) discovery of gaps in existing physical and measurements (BGP) infrastructures; ($iii$) identification of how connectivity between ASes varies in developing areas (e.g., Latin America) and across different regions of the world, and ($iv$) enhancement on knowledge about topology redundancy. We expect that our methodology will improve our understanding about Internet topology, and allow better resilience and recovery methods to mitigate disruption events.

	%by improving the geolocation inferences of peering interconnections and topology elements, events involving Internet topology that impact user experience or disrupt services~\cite{routerDMV, routerUnited, facebookCDN} will be repaired faster and more efficiently.

	\textbf{Outline of the proposal.} Chapter~\ref{cap:background} provides background and terminology. Chapter~\ref{cap:related-work} presents the state-of-the-art and the main related work. In Chapter~\ref{cap:proposal}, we present the proposal work. In Chapter~\ref{cap:methodology}, we show the expected methodology, set of steps and schedule. Chapter~\ref{cap:expected-results} provides the expected results and main contributions. Finally, in Chapter~\ref{cap:coursework}, we present the expected coursework of the PhD course.



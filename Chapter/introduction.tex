\chapter{Introduction}\label{cap:introduction}
\thispagestyle{empty}

	\textbf{Context.} Peering infrastructures, such as colocation facilities and Internet Exchange Points (IXPs) are increasingly deployed all over the world, supporting a growing number of network members and peering interconnections \cite{Giotsas:2017:DPI:3098822.3098855}. Colos are physical locations that provide essential infrastructure as power, space and supports interconnection of networks. IXPs are physical infrastructures and provide a shared switching fabric where participating networks can interconnect their routers. \cite{Giotsas:2015:MPI:2716281.2836122}

	Brazil peering infrastructure maintains the largest set of public IXPs in a single country and is among the world’s top ten IXPs in terms of traffic.~\cite{DissectingBrazilianIXP}. Currently, there are 31 public IXPs deployed in all regions~\cite{IXbr}. Access to privileged data (i.e., flows samples and BGP information) of the major IXPs of the IX.br Project (e.g., IX-SP and IX-RS) leaves large room to explore measurement studies. 

	\textbf{Motivation.} Recent methodologies \cite{Giotsas:2015:MPI:2716281.2836122} provideaccurate results but can not scale to large scenarios with several coloc, IXPs and ASes, given the large amount of active probing campaigns needed. The initial process of mapping networks, and IXPs to facilities is manual and time-consuming to be developed/updated. The work of \cite{Giotsas:2017:DPI:3098822.3098855} combines BGP communities with a colocation map to infer the location of outages. However, the map is done similarly as in \cite{Giotsas:2015:MPI:2716281.2836122}. Besides, BGP communities have no standardized semantics and are only employed by half of the BGP paths announced on the Internet.

	Expand existing results would provide a more scalable and more accurate interconnection mapping which would help to pinpoint outages and attacks, troubleshoot network problems, track traffic flows and improve the resilience of interconnections more precisely.

	Investigate if leveraging IXPs as scalable vantage points, performing measurements campaigns from IXP to the outside Internet, and using available privileged data (i.e., flows samples and BGP information) can improve the interconnection mapping to facilities and create a broader, more accurate and scalable methodology which would be used to enhance network infrastructure and safety. 
	Looking Glass has some servers colocated with IXPs that can be used as vantage points. However, they are not scalable since they are not appropriate for scanning a large range of addresses due to probing limitations~\cite{Giotsas:2015:MPI:2716281.2836122}.


	\textbf{Expected contributions.} We expect that leveraging IXPs as scalable vantage points, given their global role seeing traffic from a large fraction of the Internet~\cite{Chatzis:2013:BUL:2504730.2504746}, and using available privileged data (i.e., flows samples and BGP information) will enable higher visibility and knowledge about the geolocalization of peering interconnections, and will allow creating a broader and more accurate interconnection mapping to facilities. 

	Having a precise and scalable methodology to map peering interconnections to facilities could improve the understanding of interconnections efficiency (interconnection in different facilities in the same area could decrease the number of hops or latency) and level of peering interconnection redundancy.


	\textbf{Outline of the proposal.} Chapter~\ref{cap:background} provides background and terminology. Chapter~\ref{cap:related-work} presents the state-of-the-art and the main related work. In Chapter~\ref{cap:proposal}, we present the proposal work. Chapter~\ref{cap:expected-results} provides the expected results and main contributions. In Chapter~\ref{cap:methodology}, we show the expected methodology, set of steps and schedule. Finally, in Chapter~\ref{cap:coursework}, we present the expected coursework of the Ph.D. course.



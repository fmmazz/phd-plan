\chapter{Methodology}\label{cap:methodology}
\thispagestyle{empty}

We now present the methodology, schedule, and collaboration of the proposed work. First, we outline the main steps needed to develop the research described in this plan (Section~\ref{sec:steps}). Next, in Section~\ref{sec:schedule}, we detail a proposed schedule for the entire Ph.D. period, composed of actions and their expected duration. Finally, we conclude describing the intended collaboration with one of the leading research groups in the topic of the Ph.D. plan (Section~\ref{sec:collaboration}). 

\section{Steps}
\label{sec:steps}

\begin{enumerate}
\item {\bf Monitoring and study of state-of-the-art}: we will perform an in-depth state-of-the-art study and monitoring of themes related to the work during all the duration of the Ph.D. Also, in the initial period of the course, we will conduct a detailed examination and reproduction of the main related existing methodologies.

\item {\bf Methodology modeling}: in this step, we will develop a systematic methodology using IXPs as vantage points to map and produce geolocation inferences, seeking to answer the proposed research questions. Also at this stage, we will analyze the potential data to be used and design the data collection campaigns and data preparation.

\item {\bf Data collection}: in this phase, we will perform the data collection. At this stage, we will execute active measurements campaigns and collect data already available from different sources. More specifically, we will perform active measurements from inside of the IXP to obtain a preliminary geolocation knowledge about the Internet topology. Next, we will gather already collected information on control and data planes (e.g., BGP) and correlate with the active measurement data previously obtained to increase the accuracy of the generated inferences.

\item {\bf Data preparation}: in this point, the collected data will be preprocessed and prepared, including the combination of data from different sources, to serve as input for the proposed methodology. The collection of control plane information (e.g., BGP) and measurements campaigns tends to generate a significant amount of data. In order to process all data efficiently and without imposing resource and performance overheads to the IXP, we plan to use cloud environments (e.g., Azure) capable of dealing with a massive volume of data.

\item {\bf Methodology evaluation}: in this step, we will use de collected and preprocessed data to evaluate and validate the effectiveness of the developed methodology, identifying its features and limitations.

\item {\bf Methodology validation}: the final step of the study, we will validate the obtained results of the proposed methodology. For the validation, we plan to contact network operators and IXPs, use privileged data from inside the IXPs as flow samples, make use of ground-truth datasets and include other sources of information to improve the verification of our results. 

%\item {\bf Preliminary modeling}: in this step, we will model a methodology capable of using IXPs as scalable vantage points and improve interconnection mapping to facilities;

%\item {\bf Preliminary evaluation}: preliminary evaluation of the previously proposed methodology, using small-scale measurements campaigns and a subset of the available data to obtain validation.;

%\item {\bf Prototype development}: in this step, a prototype of the proposed methodology will be developed to perform the experimental evaluation. The objective will be to create a scalable product to be used in IXPs environments;

%\item {\bf Experimental Evaluation}: final step of the study, including large-scale evaluation, composed of campaigns of measurements and data collection, to evaluate and validate the effectiveness of the developed methodology.

\end{enumerate}

\section{Proposed schedule}
\label{sec:schedule}

\begin{enumerate}

\item In-depth state-of-the-art study and monitoring about themes related with thesis; \label{it:estado-da-arte}

\item Development of a systematic methodology using IXPs as vantage points to map and produce geolocation inferences; \label{it:modelagem}

\item Qualification Exam; \label{it:qualif}

\item Examination of proficiency in English; \label{it:english}

\item Examination of proficiency in a foreign language; \label{it:idioma}

\item Data collection and preprocessing \label{it:aval-anali}

\item Period reserved for doctorate sandwich; \label{it:sanduba}

\item Evaluation, review and reassessment of the proposed methodology; \label{it:aval-larga-escala}

\item Validation the obtained results of the proposed methodology; \label{it:proj-amb-exp}

\item Thesis Proposal Defense; \label{it:def-prop-tese}

\item Improvement of the methodology considering the results obtained in previous activities, also considering contributions and recommendations of the evaluation committee of the thesis proposal; \label{it:rev-final-modelo}

\item Thesis writing; \label{it:redacao}

\item Thesis defense; \label{it:def-tese}

\item Participation in conferences and symposia related to the theme of the thesis; \label{it:part-cong}

\item Writing and submitting articles for conferences and periodicals based on the results obtained during the studies and evaluations, related to the theme of this doctoral proposal. \label{it:submissoes}
\end{enumerate}

\begin{table}[htp]
\centering
\begin{center}
% use packages: array,color,colortbl
\newcommand{\mc}[3]{\multicolumn{#1}{#2}{#3}}
\definecolor{tcA}{gray}{0.875}
\definecolor{tcB}{gray}{0.75}
\definecolor{tcC}{gray}{0.957}
%
\begin{tabular}{|>{\columncolor{tcA}}l|l|l|l|l|l|l|l|l|}\hline
% ----------- cabeçalho
\rowcolor{tcB}
\mc{1}{|>{\columncolor{tcB}}l|}{Activities} &
\mc{1}{|>{\columncolor{tcB}}l|}{2019/1} &
\mc{1}{|>{\columncolor{tcB}}l|}{2019/2} &
\mc{1}{|>{\columncolor{tcB}}l|}{2020/1} &
\mc{1}{|>{\columncolor{tcB}}l|}{2020/2} &
\mc{1}{|>{\columncolor{tcB}}l|}{2021/1} &
\mc{1}{|>{\columncolor{tcB}}l|}{2021/2} &
\mc{1}{|>{\columncolor{tcB}}l|}{2022/1} &
\mc{1}{|>{\columncolor{tcB}}l|}{2022/2} \\
\hline
% -------- Avaliação do estado-da-arte
\mc{1}{|>{\columncolor{tcA}}c|}{\ref{it:estado-da-arte}} &
\mc{1}{|>{\columncolor[gray]{0.3}}l|}{} &
\mc{1}{|>{\columncolor[gray]{0.3}}l|}{} &
\mc{1}{|>{\columncolor[gray]{0.3}}l|}{} &
\mc{1}{|>{\columncolor[gray]{0.3}}l|}{} &
\mc{1}{|>{\columncolor[gray]{0.3}}l|}{} &
\mc{1}{|>{\columncolor[gray]{0.3}}l|}{} &
\mc{1}{|>{\columncolor[gray]{0.3}}l|}{} &
\mc{1}{|>{\columncolor[gray]{0.3}}l|}{} \\
\hline
% -------- Modelagem inicial
\mc{1}{|>{\columncolor{tcA}}c|}{\ref{it:modelagem}} &
\mc{1}{|>{\columncolor{tcC}}l|}{ } &
\mc{1}{|>{\columncolor[gray]{0.3}}l|}{} &
\mc{1}{|>{\columncolor[gray]{0.3}}l|}{} &
\mc{1}{|>{\columncolor{tcC}}l|}{ } &
\mc{1}{|>{\columncolor{tcC}}l|}{ } &
\mc{1}{|>{\columncolor{tcC}}l|}{ } &
\mc{1}{|>{\columncolor{tcC}}l|}{ } &
\mc{1}{|>{\columncolor{tcC}}l|}{ } \\
\hline
% -------- Exame de Qualificação
%\rowcolor{tcC}
\mc{1}{|>{\columncolor{tcA}}c|}{\ref{it:qualif}} &
\mc{1}{|>{\columncolor{tcC}}l|}{ } &
\mc{1}{|>{\columncolor[gray]{0.3}}l|}{} &
\mc{1}{|>{\columncolor{tcC}}l|}{ } &
\mc{1}{|>{\columncolor{tcC}}l|}{ } &
\mc{1}{|>{\columncolor{tcC}}l|}{ } &
\mc{1}{|>{\columncolor{tcC}}l|}{ } &
\mc{1}{|>{\columncolor{tcC}}l|}{ } &
\mc{1}{|>{\columncolor{tcC}}l|}{ } \\
\hline
% -------- Exame de Proficiência em Ingles
%\rowcolor{tcC}
\mc{1}{|>{\columncolor{tcA}}c|}{\ref{it:english}} &
\mc{1}{|>{\columncolor{tcC}}l|}{ } &
\mc{1}{|>{\columncolor[gray]{0.3}}l|}{} &
\mc{1}{|>{\columncolor{tcC}}l|}{ } &
\mc{1}{|>{\columncolor{tcC}}l|}{ } &
\mc{1}{|>{\columncolor{tcC}}l|}{ } &
\mc{1}{|>{\columncolor{tcC}}l|}{ } &
\mc{1}{|>{\columncolor{tcC}}l|}{ } &
\mc{1}{|>{\columncolor{tcC}}l|}{ } \\
\hline
% -------- Exame de Proficiência em Língua Estrangeira
%\rowcolor{tcC}
\mc{1}{|>{\columncolor{tcA}}c|}{\ref{it:idioma}} &
\mc{1}{|>{\columncolor{tcC}}l|}{ } &
\mc{1}{|>{\columncolor[gray]{0.3}}l|}{} &
\mc{1}{|>{\columncolor{tcC}}l|}{ } &
\mc{1}{|>{\columncolor{tcC}}l|}{ } &
\mc{1}{|>{\columncolor{tcC}}l|}{ } &
\mc{1}{|>{\columncolor{tcC}}l|}{ } &
\mc{1}{|>{\columncolor{tcC}}l|}{ } &
\mc{1}{|>{\columncolor{tcC}}l|}{ } \\
\hline
% -------- Avaliação por métodos analíticos
%\rowcolor{tcC}
\mc{1}{|>{\columncolor{tcA}}c|}{\ref{it:aval-anali}} &
\mc{1}{|>{\columncolor{tcC}}l|}{ } &
\mc{1}{|>{\columncolor[gray]{0.3}}l|}{} &
\mc{1}{|>{\columncolor[gray]{0.3}}l|}{} &
\mc{1}{|>{\columncolor{tcC}}l|}{ } &
\mc{1}{|>{\columncolor{tcC}}l|}{ } &
\mc{1}{|>{\columncolor{tcC}}l|}{ } &
\mc{1}{|>{\columncolor{tcC}}l|}{ } &
\mc{1}{|>{\columncolor{tcC}}l|}{ } \\
\hline

% -------- Afastamento para doutorado sanduíche
%\rowcolor{tcC}
\mc{1}{|>{\columncolor{tcA}}c|}{\ref{it:sanduba}} &
\mc{1}{|>{\columncolor{tcC}}l|}{ } &
\mc{1}{|>{\columncolor{tcC}}l|}{ } &
\mc{1}{|>{\columncolor[gray]{0.3}}l|}{} &
\mc{1}{|>{\columncolor[gray]{0.3}}l|}{} &
\mc{1}{|>{\columncolor{tcC}}l|}{ } &
\mc{1}{|>{\columncolor{tcC}}l|}{ } &
\mc{1}{|>{\columncolor{tcC}}l|}{ } &
\mc{1}{|>{\columncolor{tcC}}l|}{ } \\
\hline
% -------- Revisão do Modelo
%\mc{1}{|>{\columncolor{tcA}}c|}{\ref{it:rev-modelo}} &
%\mc{1}{|>{\columncolor{tcC}}l|}{ } &
%\mc{1}{|>{\columncolor{tcC}}l|}{ } &
%\mc{1}{|>{\columncolor{tcC}}l|}{ } &
%\mc{1}{|>{\columncolor[gray]{0.3}}l|}{} &
%\mc{1}{|>{\columncolor{tcC}}l|}{ } &
%\mc{1}{|>{\columncolor{tcC}}l|}{ } &
%\mc{1}{|>{\columncolor{tcC}}l|}{ } &
%\mc{1}{|>{\columncolor{tcC}}l|}{ } \\
%\hline
% -------- Projeto de um ambiente de experimentação
%\rowcolor{tcC}
\mc{1}{|>{\columncolor{tcA}}c|}{\ref{it:aval-larga-escala}} &
\mc{1}{|>{\columncolor{tcC}}l|}{ } &
\mc{1}{|>{\columncolor{tcC}}l|}{ } &
\mc{1}{|>{\columncolor[gray]{0.3}}l|}{} &
\mc{1}{|>{\columncolor[gray]{0.3}}l|}{} &
\mc{1}{|>{\columncolor{tcC}}l|}{ } &
\mc{1}{|>{\columncolor{tcC}}l|}{ } &
\mc{1}{|>{\columncolor{tcC}}l|}{ } &
\mc{1}{|>{\columncolor{tcC}}l|}{ } \\
\hline
% -------- Avaliação experimental em larga escala
%\rowcolor{tcC}
\mc{1}{|>{\columncolor{tcA}}c|}{\ref{it:proj-amb-exp}} &
\mc{1}{|>{\columncolor{tcC}}l|}{ } &
\mc{1}{|>{\columncolor{tcC}}l|}{ } &
\mc{1}{|>{\columncolor{tcC}}l|}{ } &
\mc{1}{|>{\columncolor[gray]{0.3}}l|}{} &
\mc{1}{|>{\columncolor[gray]{0.3}}l|}{} &
\mc{1}{|>{\columncolor{tcC}}l|}{ } &
\mc{1}{|>{\columncolor{tcC}}l|}{ } &
\mc{1}{|>{\columncolor{tcC}}l|}{ } \\
\hline
% -------- Defesa da Proposta Tese
%\rowcolor{tcC}
\mc{1}{|>{\columncolor{tcA}}c|}{\ref{it:def-prop-tese}} &
\mc{1}{|>{\columncolor{tcC}}l|}{ } &
\mc{1}{|>{\columncolor{tcC}}l|}{ } &
\mc{1}{|>{\columncolor{tcC}}l|}{ } &
\mc{1}{|>{\columncolor[gray]{0.3}}l|}{} &
\mc{1}{|>{\columncolor{tcC}}l|}{ } &
\mc{1}{|>{\columncolor{tcC}}l|}{ } &
\mc{1}{|>{\columncolor{tcC}}l|}{ } &
\mc{1}{|>{\columncolor{tcC}}l|}{ } \\
\hline
% -------- Aperfeiçoamento do modelo
\mc{1}{|>{\columncolor{tcA}}c|}{\ref{it:rev-final-modelo}} &
\mc{1}{|>{\columncolor{tcC}}l|}{ } &
\mc{1}{|>{\columncolor{tcC}}l|}{ } &
\mc{1}{|>{\columncolor{tcC}}l|}{ } &
\mc{1}{|>{\columncolor{tcC}}l|}{ } &
\mc{1}{|>{\columncolor[gray]{0.3}}l|}{} &
\mc{1}{|>{\columncolor[gray]{0.3}}l|}{} &
\mc{1}{|>{\columncolor{tcC}}l|}{ } &
\mc{1}{|>{\columncolor{tcC}}l|}{ } \\
\hline
%\rowcolor{tcC}
% -------- Redação Tese
%\rowcolor{tcC}
\mc{1}{|>{\columncolor{tcA}}c|}{\ref{it:redacao}} &
\mc{1}{|>{\columncolor{tcC}}l|}{ } &
\mc{1}{|>{\columncolor{tcC}}l|}{ } &
\mc{1}{|>{\columncolor{tcC}}l|}{ } &
\mc{1}{|>{\columncolor[gray]{0.3}}l|}{} &
\mc{1}{|>{\columncolor[gray]{0.3}}l|}{} &
\mc{1}{|>{\columncolor[gray]{0.3}}l|}{} &
\mc{1}{|>{\columncolor[gray]{0.3}}l|}{} &
\mc{1}{|>{\columncolor{tcC}}l|}{ } \\
\hline
% -------- Defesa Tese
\mc{1}{|>{\columncolor{tcA}}c|}{\ref{it:def-tese}} &
\mc{1}{|>{\columncolor{tcC}}l|}{ } &
\mc{1}{|>{\columncolor{tcC}}l|}{ } &
\mc{1}{|>{\columncolor{tcC}}l|}{ } &
\mc{1}{|>{\columncolor{tcC}}l|}{ } &
\mc{1}{|>{\columncolor{tcC}}l|}{ } &
\mc{1}{|>{\columncolor{tcC}}l|}{ } &
\mc{1}{|>{\columncolor{tcC}}l|}{ } &
\mc{1}{|>{\columncolor[gray]{0.3}}l|}{} \\
\hline
 % -------- Participação Congressos
%\rowcolor{tcC}
\mc{1}{|>{\columncolor{tcA}}c|}{\ref{it:part-cong}} &
\mc{1}{|>{\columncolor[gray]{0.3}}l|}{} &
\mc{1}{|>{\columncolor[gray]{0.3}}l|}{} &
\mc{1}{|>{\columncolor[gray]{0.3}}l|}{} &
\mc{1}{|>{\columncolor[gray]{0.3}}l|}{} &
\mc{1}{|>{\columncolor[gray]{0.3}}l|}{} &
\mc{1}{|>{\columncolor[gray]{0.3}}l|}{} &
\mc{1}{|>{\columncolor[gray]{0.3}}l|}{} &
\mc{1}{|>{\columncolor[gray]{0.3}}l|}{} \\
\hline
% -------- Redação Submissão Artigos
\mc{1}{|>{\columncolor{tcA}}c|}{\ref{it:submissoes}} &
\mc{1}{|>{\columncolor[gray]{0.3}}l|}{} &
\mc{1}{|>{\columncolor[gray]{0.3}}l|}{} &
\mc{1}{|>{\columncolor[gray]{0.3}}l|}{} &
\mc{1}{|>{\columncolor[gray]{0.3}}l|}{} &
\mc{1}{|>{\columncolor[gray]{0.3}}l|}{} &
\mc{1}{|>{\columncolor[gray]{0.3}}l|}{} &
\mc{1}{|>{\columncolor[gray]{0.3}}l|}{} &
\mc{1}{|>{\columncolor[gray]{0.3}}l|}{} \\
\hline
\end{tabular}
\end{center}
\caption{Schedule of activities during the Ph.D. period}
\label{tab:planejamento-doutorado}
\end{table}

\section{Collaboration}
\label{sec:collaboration}

The proposed Ph.D. is joint work with the University of California San Diego (UCSD) in cooperation with Kimberly Claffy\footnote{\url{http://www.caida.org/~kc/}} and Bradley Huffaker\footnote{\url{http://www.caida.org/~bhuffake/}}. The collaboration aims to expand the existing relationship between both universities and presents the opportunity for a sandwich doctorate. The second year of Ph.D. is expected to be located at CAIDA/UCSD. The funding for the away period is yet to be determined.


\chapter{Proposed Work}\label{cap:proposal}
\thispagestyle{empty}

	%Investigate if leveraging IXPs as scalable vantage points, performing measure- ments campaigns from IXP to the outside Internet, and using available privileged data (i.e., flows samples and BGP information) can improve the interconnection mapping to facilities and create a broader, more accurate and scalable methodology which would be used to enhance network infrastructure and safety. Investigate if using IXPs as vantage points could improve the initial IXP/AS to facility mapping through measurements in- stead of relying on online resources (PeeringDB, IXPs/ASes websites).

	Recent studies reveal the opportunity to improve the generation of geolocation knowledge. IXPs, in this context, shows potential for the emergence of new solutions involving Internet mapping, due to their central roles in topology enable higher visibility and knowledge about the network.

	This PhD work plan seeks to investigate techniques to explore Internet Exchange Points as powerful vantage points to improve the Internet cartography and generate geolocation inferences of peering interconnections and topology elements.  We aim to produce a hybrid approach that combines the advantages of both active (i.e., accuracy) and passive (i.e., scalability) solutions while being reproducible. First, we plan to gather already collected information on control and data planes (e.g., BGP) of a single IXP to obtain a preliminary geolocation knowledge about the Internet topology. Next, we intend to perform active measurements campaigns from inside of the IXP and correlate with passive data analysis to increase the accuracy of the generated inferences. Finally, we aim to scale our methodology on several IXPs to obtain comprehensive and accurate geolocation inferences. We will contact network operators and use additional available public and private data sources (e.g., flow samples) to validate our results.

	
	Our goal is to develop a methodology that achieves \emph{accuracy}, \emph{scalability} and \emph{reproducibility} at the same time. Designing a solution with these three purposes will generate a reproducible technique able to provide reliable and correct geolocation conclusions while using few key vantage points and generating little processing and networking overheads. We intend to produce a methodology that can be used by academia and industry for the development of new research and by network operators to apply to practical situations. We plan to deploy and validate our methodology using the Brazilian IX ecosystem, to which we have access.

	First, we aim to enhance geolocation inferences in developing regions such as Latin America, which show rich, but weakly examined, peering infrastructures~\cite{IXbr, DissectingBrazilianIXP}. Next, we aim to extend our methodology to other worldwide available IXPs and develop a better understanding of the geolocation characteristics of these infrastructures in the Internet Topology.

	%\textbf{Data to be used.} We can use data sources as PeeringDB, PCH, DataCenterMap, Inflect Data Center and Peering Mapping (https://inflect.com), IXP, ASes, and Network Operating Centers (NOCs) websites and lists provided in regional consortia of IXPs, as well as active measurements using the IXP as a vantage point to build a mapping between AS, IXP, and facilities. We can use other vantage points as RIPE Atlas, Looking Glass (Periscope \cite{Periscope}), iPlane and CAIDA Ark to augment our mapping or to validate the mapping obtained from the IXP point of view. We can use datasets of BGP to leverage the BGP Communities attribute and acquire accurate location information for about half of all BGP IPv4 updates as \cite{Giotsas:2017:DPI:3098822.3098855}. We can use flow sample datasets from IX.br to investigate how many IXPs a given flow is traversing, infer and "see" from where traffic arrives, leaves, where it comes from. Also to obtain the ground truth of this IXP’s public peering fabric, map MAC addresses to router IP addresses and their respective AS numbers~\cite{Ager:2012} and information about how two parties of an IXP peering use that link and for what purpose~\cite{Richter:2014}.

	%\textbf{Assumptions.}

	%\textbf{Set of metrics.} Number of peering interfaces inferred. Fraction of resolved interfaces when dealing with less vantage points used. Fraction of unresolved interfaces with when removing facilities information. Fraction of ground truth locations that match inferred locations. \cite{Giotsas:2015:MPI:2716281.2836122}. Number of peering interfaces inferred by one IXP. Error probability of inferred location.

	\textbf{Research Questions.} In this work, we aim to answer the following research questions: 

	\begin{itemize}
	\item Can IXPs be used as vantage points to generate geolocation inferences about peering interconnections and Internet topology? 
	\item How can we obtain accurate geolocation of infrastructure elements from inside the IXP? 
	\item Is it possible to correlate active measurement data with collected information on control and data planes (e.g., BGP) to improve the Internet cartography?
	\item Which and how many IXPs are necessary to have a precise view of the Internet? 
	\item What are the implications of using this approach concerning computational cost, network traffic, and privacy? 
	\item How is it possible to measure in a scalable and automatized way?
	\end{itemize}

	\textbf{Risks and limitations.} The proposed research presents risks on data and IXP access. More specifically, data from the IXP is usually confidential and may not be available to the general public. Besides, IXPs tend to restrict access to their network. We will not face these risks because of already collected data and ongoing collaborations with the Brazilian IXP from another PhD study. Besides, our methodology can present potential limitations. We could face inaccurate results when using few IXPs given that their visibility, individually, may not perfect and our technique may not be applied in regions (e.g., US) where IXPs tend to be more restrictive. 
